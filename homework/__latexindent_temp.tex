\documentclass[boxes, qed]{homework}
\usepackage{amsmath}
\name{Rohit Wason}
\course{Math 501}
\term{Spring 2021}
\hwnum{(\S1.2)}

\newcommand{\bigzero}{\mbox{\normalfont\Large\bfseries 0}}
\newcommand{\rvline}{\hspace*{-\arraycolsep}\vline\hspace*{-\arraycolsep}}

\begin{document}

\newenvironment{amatrix}[1]{%
  \left[\begin{array}{@{}*{#1}{c}|c@{}}
}{%
  \end{array}\right]
}

\newenvironment{augmatrix}[1]{%
  \left[\begin{array}{#1}
}{%
  \end{array}\right]
}

\S1.2
\problemnumber{1}
\begin{problem}
    A pipe maker company produces 3ft and 5ft lengths of pipes. 
    These pieces can be glued together to get longer pipes. 
    Prove that any integer length of at least 8ft can be 
    made out of these pipes.
\end{problem}
\begin{solution}
    In other words we need to prove that the following equation
    has positive solutions for all $c \ge 8$:
    $$3a + 5b = c$$
    Let $P(c)$ be the statement that the last statement is true for
    a given $c$.\\

    \textbf{basis step:} $P(8)$ holds true, since $a=1, b=1$ 
    are the positive solutions.\\

    \textbf{induction step:} Let's assume $P(k)$ holds for an arbitrary
    $k>8$. I.e., $\exists a,b: a\ge{0},b\ge{0}$ and
    $$k=3a+5b$$
    Three cases arise while considering the quantity $k+1$:
    \begin{enumerate}
        \item[Case I] 
        \begin{align*}
            a\ge{1}, b\ge{1} 
                \implies k+1 = 3a + 5b + 1 
                = 3a + 5(b-1) + 6
                = 3(a+2) + 5(b-1)
        \end{align*}

        \item[Case II] $a=0, b\ge{1} 
        \implies k+1 = 5b + 1 
        = 5(b-1) + 6
        = 3(2) + 5(b-1)$

        \item[Case III] $a\ge{1}, b=0 
        \implies k+1 = 5b + 1 
        = 5(b-1) + 6
        = 3(2) + 5(b-1)$
    \end{enumerate}
    
\end{solution}
\end{document}
