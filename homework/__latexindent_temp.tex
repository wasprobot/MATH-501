\documentclass[boxes, qed]{homework}
\usepackage{amsmath}
\name{Rohit Wason}
\course{Math 501}
\term{Spring 2021}
\hwnum{(\S2.1)}

\newcommand{\bigzero}{\mbox{\normalfont\Large\bfseries 0}}
\newcommand{\rvline}{\hspace*{-\arraycolsep}\vline\hspace*{-\arraycolsep}}

\begin{document}

\newenvironment{amatrix}[1]{%
  \left[\begin{array}{@{}*{#1}{c}|c@{}}
}{%
  \end{array}\right]
}

\newenvironment{augmatrix}[1]{%
  \left[\begin{array}{#1}
}{%
  \end{array}\right]
}

\problemnumber{1}
\begin{problem}Page 27, in the proof of Theorem 2.1.8: 
  "then $a^2=(-a)(-a)$". So why is that?
\end{problem}
\begin{solution}We can take the R.H.S. $(-a)(-a)$
\end{solution}
\begin{problem}Page 28, in the proof of Theorem 2.1.10: 
  "If $a>0$, then $1/a>0$." Why is this true?
\end{problem}
\begin{solution}By the definition of reciprocals,
  $a\cdot{1/a}=1$. Since there is a theorem that
  $1>0$, $a\cdot{1/a}>0$
\end{solution}
\end{document}
