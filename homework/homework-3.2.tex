\documentclass[boxes, qed]{homework}
\usepackage{amsmath}
\name{Rohit Wason}
\course{Math 501}
\term{Spring 2021}
\hwnum{(\S3.2)}

\newcommand{\bigzero}{\mbox{\normalfont\Large\bfseries 0}}
\newcommand{\rvline}{\hspace*{-\arraycolsep}\vline\hspace*{-\arraycolsep}}

\begin{document}

\newenvironment{amatrix}[1]{%
  \left[\begin{array}{@{}*{#1}{c}|c@{}}
}{%
  \end{array}\right]
}

\newenvironment{augmatrix}[1]{%
  \left[\begin{array}{#1}
}{%
  \end{array}\right]
}
\problemnumber{15}
\begin{problem}To prove: If $0<a<b, z_n := (a^n+b^n)^{1/n} \rightarrow b$
\end{problem}
\begin{solution}We first claim that $(2^{1/n})$ converges to $1$.
  We can prove this by observing that
  $$\log_{2}(2^{1/n}) = \frac{1}{n}\log_{2}(2) = \frac{1}{n}$$
  $$\therefore 2^{1/n} \rightarrow 2^0 = 1$$
  since $1/n$ converges to $0$.
  Now since
  \begin{align*}
    0 < a < b\\
    0 < a^n < b^n\\
    b^n < a^n+b^n < 2b^n\\
    (b^n)^{1/n} < (a^n+b^n)^{1/n} < (2b^n)^{1/n}\\
    b < z_n < 2^{1/n}b\\
    b \le z_n \le 2^{1/n}b\\
    \text{ (by laxing the inequality)}
  \end{align*}
  Since we know that $(2^{1/n}) \rightarrow 1$ 
  and $(b) \rightarrow b$, using Squeeze Theorem,
  we have that $z_n \rightarrow b$.
\end{solution}
\end{document}
