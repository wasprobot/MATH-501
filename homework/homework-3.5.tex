\documentclass[boxes, qed]{homework}
\usepackage{amsmath}
\usepackage{xcolor}
\usepackage{listings}
\name{Rohit Wason}
\course{Math 501}
\term{Spring 2021}
\hwnum{(\S3.5)}

\newcommand{\bigzero}{\mbox{\normalfont\Large\bfseries 0}}
\newcommand{\rvline}{\hspace*{-\arraycolsep}\vline\hspace*{-\arraycolsep}}
\DeclarePairedDelimiter\abs{\lvert}{\rvert}
\DeclarePairedDelimiter\norm{\lVert}{\rVert}

\begin{document}

\newenvironment{amatrix}[1]{%
  \left[\begin{array}{@{}*{#1}{c}|c@{}}
}{%
  \end{array}\right]
}

\newenvironment{augmatrix}[1]{%
  \left[\begin{array}{#1}
}{%
  \end{array}\right]
}
\problemnumber{15}
\begin{problem}Approximate the root $0<r<1$ of equation
  $x^3-5x+1=0$ within $10^{-4}$.
\end{problem}
\begin{solution}We first rewrite the equation as $x=\frac{1}{5}(x^3+1)$.
  Then we pick an arbitrary $0<x_1<1$ and define
  \begin{equation}
    x_{n+1}:=\frac{1}{5}(x_n^3+1)
  \end{equation}

  Let's use Induction to show that $0<x_n<1$ for all $n$. With the base clear,
  and assuming a particular $0<x_n<1$, we can see that $1<x_n^3+1<2$
  and $1/5<(x_n^3+1)<2/5$. Thus using (15.1), since $0<x_{n+1}=x_n^3+1$
  \begin{equation}
    0<x_n<1
  \end{equation}

  Now let's look at the difference in subsequent terms of $x_n$
  \begin{align*}
      &\abs[\Big]{x_{n+2}-x_{n+1}}\\
      &=\abs[\Big]{
        \frac{1}{5}(x_{n+1}^3+1)
        -\frac{1}{5}(x_{n}^3+1)
      }
      =\frac{1}{5}
      \abs[\Big]{x_{n+1}^3 - x_{n}^3}\\
      &=\frac{1}{5}
      \abs[\Big]{x_{n+1}^2 + x_{n+1}x_{n} + x_{n}^2}
      \abs[\Big]{x_{n+1}- x_{n}}\\
      &\le\frac{3}{5}\abs[\Big]{x_{n+1}- x_{n}}
      &\text{(since $0<x_n<1$)}
  \end{align*}

  This implies that $X=(x_n)$ is a \textbf{contractive sequence} with $C=\frac{3}{5}$
  and that $\lim{X}=r$ exists. Substituting $r$ in $(15.1)$ we get
  $r=\frac{1}{5}(r^3+1)$ or $r^3-5r+1=0$. Hence $r$ is a root of the given equation.
  Now to approximate the value of $r$ we can
  \begin{enumerate}
    \item Pick $x_1=0.5$
    \item Calculate $x_2=0.225$
    \item Calculate $x_3=0.202278125$
    \item \dots
  \end{enumerate}
  
  In order to know how many such iterations we need, we use Corollary $3.5.10(i)$
  from the book which states that for a contractive sequence with limit, $r$ and
  contractive constant, $C=\frac{3}{5}$ here,
  \begin{align*}
    \abs[\Big]{r-x_n} 
      &\le \frac{C^{n-1}}{1-C} \abs[Big]{x_2-x_1}\\
      &\le \frac{3^{n-1}}{5^{n-2}} (\frac{0.275}{2})\\
      &\le \frac{3^{n-1}}{5^{n-2}} (0.1375)
  \end{align*}
  This inequality tells  us where to stop, depending on the precision we seek in the value
  of $r$. E.g., for the required precision we know that $\abs{r-x_n} \le 10^{-4}$. 
  This means the following equation applies
  \begin{align*}
    \frac{3^{n-1}}{5^{n-2}} &= 10^{-4}\\
    (\frac{3}{5})^{n-1} &= 5 \times 10^{-4}\\
    (n-1)\ln{\frac{3}{5}} &= \ln{5} -4 \ln{10}\\
    (n-1) -0.51082562376 &= -7.60090245954\\
    n &\approx 16
  \end{align*}
  We can iterate manually, or use a computer program to calculate 
  the desired approximation of $r$:
  \begin{lstlisting}[backgroundcolor = \color{lightgray},language = R]
  x=0.5; for(n in (2:16)) x=(x^3+1)/5; print(x)
  0.2016397
  \end{lstlisting}
\end{solution}
\end{document}
