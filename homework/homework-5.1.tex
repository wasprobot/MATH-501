\documentclass[boxes, qed]{homework}
\usepackage{amsmath}
\usepackage{xcolor}
\usepackage{listings}
\name{Rohit Wason}
\course{Math 501}
\term{Spring 2021}
\hwnum{(\S5.1, Continuous Functions)}

\newcommand{\bigzero}{\mbox{\normalfont\Large\bfseries 0}}
\newcommand{\rvline}{\hspace*{-\arraycolsep}\vline\hspace*{-\arraycolsep}}
\DeclarePairedDelimiter\abs{\lvert}{\rvert}
\DeclarePairedDelimiter\norm{\lVert}{\rVert}

\begin{document}

\newenvironment{amatrix}[1]{%
  \left[\begin{array}{@{}*{#1}{c}|c@{}}
}{%
  \end{array}\right]
}

\newenvironment{augmatrix}[1]{%
  \left[\begin{array}{#1}
}{%
  \end{array}\right]
}
\problemnumber{14}
\begin{problem}Given $A:=(0,\infty)$ and $k:A\to \mathbb{R}$
  defined by
  \[\begin{cases} 
    0 & x \text{ irrational},\\
    n & x \text{ rational of the form } $m/n$, \gcd{(m,n)}=1
  \end{cases}\]
  Prove that $k$ is unbounded on every open interval in $A$
  and that it is not continuous anywhere in $A$.
\end{problem}
\begin{solution}Let $I:=(a,b)\subseteq{A}$ be an open interval. We find any two rationals 
  $$p=\frac{m_1}{n_1}, q=\frac{m_2}{n_2} \in I$$ 
  such that $\gcd{(n_1,n_2)}=1$. This is possible since, according to Density,
  there are infinitely many rationals in $I$. We take the average of $p,q$ as
  $$t=\frac{m_3}{n_1n_2}$$
  Notice that $k(t)=n_1n_2$, which is greater than both $k(p)=n_1$ and $k(q)=n_2$.
  In other words, no matter how large $k(p)$ or $k(q)$, 
  we can find a larger $k(t)$ satisfying $p,q,t\in{I}$.\\
  
  Since $I$ is arbitrary, we conclude that $k$ is \textbf{unbounded} 
  on every open interval in $A$ – \textit{including} $A$ itself.
  This means for any $M>0$ it is possible to find $x_M\in{A}$
  satisfying $|k(x_M)|>M$. In other words, $k$ is \textbf{discountinuous
  everywhere} on $A$.
\end{solution}
\end{document}
