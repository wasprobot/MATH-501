\documentclass[boxes, qed]{homework}
\usepackage{amsmath}
\usepackage{xcolor}
\usepackage{listings}
\name{Rohit Wason}
\course{Math 501}
\term{Spring 2021}
\hwnum{(\S6.1, Derivatives)}

\newcommand{\bigzero}{\mbox{\normalfont\Large\bfseries 0}}
\newcommand{\rvline}{\hspace*{-\arraycolsep}\vline\hspace*{-\arraycolsep}}
\DeclarePairedDelimiter\abs{\lvert}{\rvert}
\DeclarePairedDelimiter\norm{\lVert}{\rVert}

\begin{document}

\newenvironment{amatrix}[1]{%
  \left[\begin{array}{@{}*{#1}{c}|c@{}}
}{%
  \end{array}\right]
}

\newenvironment{augmatrix}[1]{%
  \left[\begin{array}{#1}
}{%
  \end{array}\right]
}
\problemnumber{12}
\begin{problem}If $r>0$ is rational and $f:\mathbb{R} \to \mathbb{R}$:
  $$f(x):= \begin{cases}
    x^r\sin(1/x), & x\ne{0}\\
    0, & x=0.
  \end{cases}
  $$
  Determine those values of $r$ for which $f'(0)$ exists.
\end{problem}
\begin{solution}First consider $x^s, s=m/n \in \mathbb{Q}$.\\
  When $x<0$, $\lim_{x\to{0^{-}}}x^s=0$
  for $n$ odd, but does not exist for $n$ even
  (\textbf{even roots} of negative numbers aren't real).
  Therefore
  \begin{equation}
    \displaystyle{\lim_{x\to{0}}x^s}=
    \begin{cases}
      0, & n \text{ odd}\\
      DNE, & n \text{ even.}
    \end{cases}
  \end{equation}
  Now for $f'(0)$ to exist,
  $
    \lim_{x\to{0}}\frac{f(x)-f(0)}{x-0}
    =\lim_{x\to{0}}\frac{f(x)}{x}
  $
  must exist (definition of Derivative).\\

  For $x \ne 0$, using $-1 \le \sin(1/x) \le 1$ we can write
  % $-\abs{x^r} \le f(x) \le \abs{x^r}$, 
  $-\abs{x^{r-1}} \le f(x)/x \le \abs{x^{r-1}}$, or
  \begin{equation}
    -\abs{g(x)} \le \frac{f(x)}{x} \le \abs{g(x)}, \text{for } g(x):=x^{r-1}
  \end{equation}

  \textbf{\textit{Case I:} $0<r<1$.}\\
  Let $s=1-r>0$, or $g(x)=x^s, s=m/n \in \mathbb{Q}$.
  We know from $(12.1)$ that $\lim_{x\to{0}}g(x)=0$ 
  is defined \textbf{only when $n$ is odd.}\\

  \textbf{\textit{Case II:} $r \ge 1$.}\\
  Let $s=r-1>0$, and $g(x)=x^s, s=m/n \in \mathbb{Q}$.
  As above, $\lim_{x\to{0}}g(x)=0$ 
  is defined \textbf{only when $n$ is odd.}\\

  Hence all $r=m/n,$ $n$ odd, 
  lead to $\lim_{x\to{0}}\abs{g(x)}
    = -\lim_{x\to{0}}\abs{g(x)}$.
  Using \textbf{Squeeze Theorem} and $(12.2)$ 
  this is a sufficient condition for
  $\displaystyle{\lim_{x\to{0}}\frac{f(x)}{x}}=0$
  hence the existence of $f'(0)$.
\end{solution}
\end{document}
