\documentclass[boxes, qed]{homework}
\usepackage{amsmath}
\name{Rohit Wason}
\course{Math 501}
\term{Spring 2021}
\hwnum{(\S1.3)}

\newcommand{\bigzero}{\mbox{\normalfont\Large\bfseries 0}}
\newcommand{\rvline}{\hspace*{-\arraycolsep}\vline\hspace*{-\arraycolsep}}

\begin{document}

\newenvironment{amatrix}[1]{%
  \left[\begin{array}{@{}*{#1}{c}|c@{}}
}{%
  \end{array}\right]
}

\newenvironment{augmatrix}[1]{%
  \left[\begin{array}{#1}
}{%
  \end{array}\right]
}

\S1.3
\problemnumber{1}
\begin{problem}
  Let $A$ be the set of positive integers. Let $A_i$ be the set of $i$-element subsets of $A$ $(i=0,1,...)$.
  Then let $B$ be the union of the sets $A_0, A_1, A_2, ...$
  Which of the sets $A_0, A_1, A_2,...$ and $B$ are countable?
\end{problem}
\begin{solution}
  $A_0=\{\phi\}$, and has $1$ element. It is therefore \textit{countable}.\\
  $A_1=\{\{1\},\{2\},\dots\}$. There seems to be a surjection from $\mathbb{N}$
  onto $A_1$. Since $\mathbb{N}$ is \textit{countable}, so is $A_1$.\\
  $A_2$ is nothing but $\mathbb{N}\times\mathbb{N}$ and can be shown as 
  \textit{countable} with the "diagonal procedure".\\
  The higher sets $A_3,A_4,\dots$ can be shown as
  cross products of the previous set with $A_1$ 
  e.g., $A_3=A_2\times{A_1}$, $A_4=A_3\times{A1}$ and so on.
  Since the cross product of $2$ countable sets is also countable,
  all higher $A_i$ are \textit{countable}.\\

  In the end, $B=\bigcup_{i=0}^{\infty}A_i$ being a union
  of all countable sets, is also \textit{countable} because
  intuitively it is just an additive process of counting.
\end{solution}
\end{document}
