\documentclass[boxes, qed]{homework}
\usepackage{amsmath}
\name{Rohit Wason}
\course{Math 501}
\term{Spring 2021}
\hwnum{(ternary representation)}

\newcommand{\bigzero}{\mbox{\normalfont\Large\bfseries 0}}
\newcommand{\rvline}{\hspace*{-\arraycolsep}\vline\hspace*{-\arraycolsep}}

\begin{document}

\newenvironment{amatrix}[1]{%
  \left[\begin{array}{@{}*{#1}{c}|c@{}}
}{%
  \end{array}\right]
}

\newenvironment{augmatrix}[1]{%
  \left[\begin{array}{#1}
}{%
  \end{array}\right]
}

\begin{problem}Define $[0,1]\supseteq C:=
  \{
    x:x=(.c_1c_2\dots{c_n}\dots)_3, 
    c_i\in\{0,2\},
    \forall{i\in{\mathbb{N}}}
  \}$.
\end{problem}
\begin{solution}a) To see if $C$ is countable
  let's look at another set,
  $C':=\{
    x:x=(.c_1c_2\dots{c_n}\dots)_3,
    c_i=1,
    \forall{i\in{\mathbb{N}}}
  \}$.
  All the digits in $x$ are $1$'s. Since there's only one such
  number $\frac{1}{2}=(.1111111\dots)_3$, $C'$ is \textit{countable}.
  Since $C\cup{C'} = [0,1]$ is uncountable, $C$ is 
  \textbf{uncountable}?\\

  b) Assuming there is an interval $(a, b)$ with length $l>0$
  such that $(a, b)\subseteq{C}$. By definition, 
  the subinterval $(a+\frac{l}{3},a+\frac{2l}{3})\not\subset{C}$.
  Therefore such $(a,b)$ cannot exist.
\end{solution}
\end{document}
