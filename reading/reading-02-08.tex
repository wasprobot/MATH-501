\documentclass[boxes, qed]{homework}
\usepackage{amsmath}
\usepackage{mathtools}
\name{Rohit Wason}
\course{Math 501}
\term{Spring 2021}
\hwnum{(\S{3.2} Limit Theorems)}

\newcommand{\bigzero}{\mbox{\normalfont\Large\bfseries 0}}
\newcommand{\rvline}{\hspace*{-\arraycolsep}\vline\hspace*{-\arraycolsep}}

\DeclarePairedDelimiter\abs{\lvert}{\rvert}%
\DeclarePairedDelimiter\norm{\lVert}{\rVert}%

\begin{document}

\newenvironment{amatrix}[1]{%
  \left[\begin{array}{@{}*{#1}{c}|c@{}}
}{%
  \end{array}\right]
}

\newenvironment{augmatrix}[1]{%
  \left[\begin{array}{#1}
}{%
  \end{array}\right]
}

\begin{problem}Assume $X=(x_n)\rightarrow{x}$ and
  $Z=(z_n)\rightarrow{z}, z\ne{0}$.
\end{problem}
\begin{solution}Two cases arise:
  \begin{enumerate}
    \item[Case I:] There are finitely many $z_i=0$. This implies 
      there is a $k\in{\mathbb{N}}$ after which $z_i\ne{0}$.
      In other words, $Z'=(z_n), n>k$ is a tail of $Z$ that does 
      converge to $z$. Hence the quotient sequence $X/Z$ converges.
      to $x/z$.
    \item[Case II:] There are infinetely many zero- and nonzero-elements
      in $Z$. This means no matter how large a $k\in{\mathbb{N}}$,
      we can always find a $z_{i>k}=0$. I.e., $z_n$ cannot converge.
      In this case the quotient sequence $X/Z$ does not exist.
  \end{enumerate}
\end{solution}
\end{document}
