\documentclass[boxes, qed]{homework}
\usepackage{amsmath}
\usepackage{mathtools}
\name{Rohit Wason}
\course{Math 501}
\term{Spring 2021}
\hwnum{(\S{3.3} Monotone Convergence Theorem)}

\newcommand{\bigzero}{\mbox{\normalfont\Large\bfseries 0}}
\newcommand{\rvline}{\hspace*{-\arraycolsep}\vline\hspace*{-\arraycolsep}}

\DeclarePairedDelimiter\abs{\lvert}{\rvert}%
\DeclarePairedDelimiter\norm{\lVert}{\rVert}%

\begin{document}

\newenvironment{amatrix}[1]{%
  \left[\begin{array}{@{}*{#1}{c}|c@{}}
}{%
  \end{array}\right]
}

\newenvironment{augmatrix}[1]{%
  \left[\begin{array}{#1}
}{%
  \end{array}\right]
}

\begin{problem}If $(b_n)$ is a bounded sequence, and $lim(a_n)=0$,
  then $lim(a_nb_n)=0$.
\end{problem}
\begin{solution}The argument $|m||a_n|=|ma_n|\le|a_n{b_n}|\le|M a_n|=|M||a_n|$
  is valid and using \textbf{Squeeze Theorem} leads to the conclusion $lim|a_n{b_n}|=0$.\\

  However this does not imply that $lim(a_n{b_n})=0$ (only the converse is known to be a theorem).\\

  By definition for every $\epsilon>0$ there exists $K\in{\mathbb{N}}$ such that
  when $n\ge{K}$, $||a_nb_n|-0|<\epsilon$.
  Considering both cases:
  \begin{enumerate}
    \item[Case I]: $a_n{b_n}\ge{0}
    \implies |a_n{b_n}| = a_n{b_n}$, and\\
    $lim(a_nb_n)=0$ follows from above.
    \item[Case II]: $a_n{b_n}<{0}
    \implies |a_n{b_n}| = -a_n{b_n}$, therefore\\
    $||a_nb_n|-0|=|-a_nb_n-0|<\epsilon$\\
    ???
  \end{enumerate}
\end{solution}
\end{document}
