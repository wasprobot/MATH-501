\documentclass[boxes, qed]{homework}
\usepackage{amsmath}
\usepackage{mathtools}
\name{Rohit Wason}
\course{Math 501}
\term{Spring 2021}
\hwnum{(\S{3.3} Monotone Convergence Theorem)}

\newcommand{\bigzero}{\mbox{\normalfont\Large\bfseries 0}}
\newcommand{\rvline}{\hspace*{-\arraycolsep}\vline\hspace*{-\arraycolsep}}

\DeclarePairedDelimiter\abs{\lvert}{\rvert}%
\DeclarePairedDelimiter\norm{\lVert}{\rVert}%

\begin{document}

\newenvironment{amatrix}[1]{%
  \left[\begin{array}{@{}*{#1}{c}|c@{}}
}{%
  \end{array}\right]
}

\newenvironment{augmatrix}[1]{%
  \left[\begin{array}{#1}
}{%
  \end{array}\right]
}

\begin{problem}Examine whether $(x_n)$ converges, where
  $x_1=a>0$ and $x_{n+1}=x_n+\frac{1}{x_n}$.
\end{problem}
\begin{solution}Since $x_n>0, \frac{1}{x_n}>0$. It is easy to see that
  $$x_1<x_2<\dots<x_n, n\in{\mathbb{N}}$$
  Assume the sequence had a limit, $x$. Since $(x_{n+1})$ is a 1-tail
  of $(x_n)$, they should both converge to the same limit. In order words
  $$lim(x_{n+1})=lim(x_n)=x$$
  Since the two sequences are linearly related
  their limits would also follow the same relation,
  $$x=x+\frac{1}{x}$$
  Since this leads to an absurdity $\frac{1}{x}=0$,
  our assumption that $(x_n)$ converges must be \textbf{incorrect}.
\end{solution}
\end{document}
