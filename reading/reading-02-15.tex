\documentclass[boxes, qed]{homework}
\usepackage{amsmath}
\usepackage{mathtools}
\name{Rohit Wason}
\course{Math 501}
\term{Spring 2021}
\hwnum{(\S{3.4} Subsequences)}

\newcommand{\bigzero}{\mbox{\normalfont\Large\bfseries 0}}
\newcommand{\rvline}{\hspace*{-\arraycolsep}\vline\hspace*{-\arraycolsep}}

\DeclarePairedDelimiter\abs{\lvert}{\rvert}%
\DeclarePairedDelimiter\norm{\lVert}{\rVert}%

\begin{document}

\newenvironment{amatrix}[1]{%
  \left[\begin{array}{@{}*{#1}{c}|c@{}}
}{%
  \end{array}\right]
}

\newenvironment{augmatrix}[1]{%
  \left[\begin{array}{#1}
}{%
  \end{array}\right]
}

\begin{problem}Prove if $\lim{a_n} = L_1 \neq L_2 = \lim{b_n}$, 
  then their `merge', $(c_n)$ diverges and converges if $L_1=L_2$.
\end{problem}
\begin{solution}Let's assume $c_n$ converges with 
  $L_1 \neq L_2$. W.L.O.G., let $L_2>L_1$. 
  By definition of limits, $a_n \nrightarrow L_2$.
  Since $a_n$ is a subsequence of $c_n$, by theorem $3.4.2$,
  $c_n \nrightarrow L_2$. Also by the same theorem since $b_n \rightarrow L_2$
  it cannot be a subsequence of $c_n$ which is against the premise. 
  Hence we must conclude that $c_n$ diverges.\\
  If, on the contrary, $L_1=L_2$, by definition of the `merge', 
  $c_n$ is bounded by $L_1$, and every subsequence of $c_n$ converges to $L_1$.
  Hence by theorem $3.4.9$, $c_n \rightarrow L_1$.
\end{solution}
\end{document}
