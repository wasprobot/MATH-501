\documentclass[boxes, qed]{homework}
\usepackage{amsmath}
\usepackage{mathtools}
\name{Rohit Wason}
\course{Math 501}
\term{Spring 2021}
\hwnum{(\S{4.3} Limits at Infinity)}

\newcommand{\bigzero}{\mbox{\normalfont\Large\bfseries 0}}
\newcommand{\rvline}{\hspace*{-\arraycolsep}\vline\hspace*{-\arraycolsep}}

\DeclarePairedDelimiter\abs{\lvert}{\rvert}%
\DeclarePairedDelimiter\norm{\lVert}{\rVert}%

\begin{document}
\begin{problem}Let $f:\mathbb{R}^+ \rightarrow \mathbb{R}$. Let $f_n=f(n)$ for $n = 1, 2, \dots$ 
  a sequence corresponding to the function $f$. Which of the following statements are true?
  \begin{enumerate}
    \item If $\lim_{x \to \infty} f(x)=L$, then $\lim(f_n)=L$.
    \item If $\lim(f_n)=L$, then $\lim_{x \to \infty} f(x)=L$
  \end{enumerate}
\end{problem}
\begin{solution}
  \begin{enumerate}
    \item Let $\lim_{x\to \infty}f(x)=L$. By definition this implies
    that for every sequence $(x_n)$ in a subset $(a,\infty)$ of $\mathbb{R}^+$,
    such that $(x_n)\to \infty$, $(f(x_n))\to L$. We can use this property to pick
    $(x_n)=\mathbb{N}$ which we know tends to $\infty$, hence $(f(x_n))\to L$.
    \textbf{TRUE}.
    
    \item Conversely, let $(f(x_n))\to L$. Again by definition every convergent
    sequence $f_n$ converges to the same limit as the function itself, hence
    $f(x)$ must converge to $L$. \textbf{TRUE}
  \end{enumerate}
\end{solution}
\end{document}
