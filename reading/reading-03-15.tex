\documentclass[boxes, qed]{homework}
\usepackage{amsmath}
\usepackage{mathtools}
\name{Rohit Wason}
\course{Math 501}
\term{Spring 2021}
\hwnum{(\S{5.1} Continuous Functions)}

\newcommand{\bigzero}{\mbox{\normalfont\Large\bfseries 0}}
\newcommand{\rvline}{\hspace*{-\arraycolsep}\vline\hspace*{-\arraycolsep}}

\DeclarePairedDelimiter\abs{\lvert}{\rvert}%
\DeclarePairedDelimiter\norm{\lVert}{\rVert}%

\begin{document}
\begin{problem}Hypothesis: Let $f:\mathbb{R}^+\to\mathbb{R}$. For every $\epsilon>0$ there
  exists $\delta>0$ such that $|x-y|<\delta$ and $x,y>0$ implies $|f(x)-f(y)|<\epsilon$.
\end{problem}
\begin{solution}Since $x,y\in\mathbb{R}^+$ are arbitrary, the given hypothesis is true
  for any point $x$ in, and every corresponding point $y$ in $\mathbb{R}^+$. 
  Hence, by definition, $f$ \textbf{is continuous} on the set $\mathbb{R}^+$.\\

  Conversely, let $f$ be continuous on every point of $\mathbb{R}^+$. 
  Let $y\in\mathbb{R}^+$ be one such point. By definition, for every $\epsilon>0$
  there exists a $\delta>0$ such that $|x-y|<\delta$ implies $|f(x)-f(y)|<\epsilon$.
  Since $x,y,>0$ this implication \textbf{is} the hypothesis.
\end{solution}
\end{document}
