\documentclass[boxes, qed]{homework}
\usepackage{enumitem,amsmath,mathtools}
\usepackage{fontspec}
\usepackage{euler}

\defaultfontfeatures{Ligatures=TeX}
% \setmainfont[Ligatures=TeX,Scale=1.25]{[FountainPen.ttf]}

\usepackage{titlesec}

\name{Rohit Wason}
\course{Math 501}
\term{Spring 2021}
\hwnum{(\S{6.1} Differentials)}

\newcommand{\bigzero}{\mbox{\normalfont\Large\bfseries 0}}
\newcommand{\rvline}{\hspace*{-\arraycolsep}\vline\hspace*{-\arraycolsep}}

\DeclarePairedDelimiter\abs{\lvert}{\rvert}%
\DeclarePairedDelimiter\norm{\lVert}{\rVert}%

\begin{document}
\begin{problem}Are there functions $f,g$ (defined on the same interval) such that neither 
  of them is constant, and $(fg)'= f'g'$.
\end{problem}
\begin{solution}For the given condition to be true,
  \begin{align*}
    (fg)'&=f'g'\\
      fg'+gf'&=f'g'
    &\text{(Product Rule)}\\
    \frac{f}{f'}+\frac{g}{g'}&=1
  \end{align*}
  Intuitively, we see that $f,g$ could be some exponent functions
  such that their ratios with their respective derivatives are constants
  that add up to $1$.\\
  In particular, choose $f=e^{3x/2}$ and $g=e^{3x}$. Using the Chain Rule
  we see that $f'=\frac{3}{2}f$ and $g'=3g$ and hence
  $$\frac{f}{f'}+\frac{g}{g'}=1$$ or the required condition is satisfied.
\end{solution}
\end{document}
