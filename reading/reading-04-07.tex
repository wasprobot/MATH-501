\documentclass[boxes, qed]{homework}
\usepackage{enumitem,amsmath,mathtools}
\usepackage{fontspec}
\usepackage{euler}

\defaultfontfeatures{Ligatures=TeX}
% \setmainfont[Ligatures=TeX,Scale=1.25]{[FountainPen.ttf]}

\usepackage{titlesec}

\name{Rohit Wason}
\course{Math 501}
\term{Spring 2021}
\hwnum{(\S{6.2} MVT)}

\newcommand{\bigzero}{\mbox{\normalfont\Large\bfseries 0}}
\newcommand{\rvline}{\hspace*{-\arraycolsep}\vline\hspace*{-\arraycolsep}}

\DeclarePairedDelimiter\abs{\lvert}{\rvert}%
\DeclarePairedDelimiter\norm{\lVert}{\rVert}%

\begin{document}
\begin{problem}Prove that the function that has a strictly positive derivative
  on an interval is strictly increasing.
\end{problem}
\begin{solution}Let $f:I\to\mathbb{R}$ have a derivative in $I$, $f'(x)>0$.
  Let $x_1,x_2\in{I}$ satisfy $x_1<x_2$. Applying the Mean Value Theorem
  to $f$ on $[x_1,x_2]$ we find $c\in{(x_1,x_2)}$ such that
  $$f(x_2)-f(x_1)=f'(c)(x_2-x_1).$$
  Since $f'(c)>0$ by hypothesis, and $x_2-x_1>0$, we have that
  $$f(x_2)-f(x_1)>0.$$
  In other words, $x_2>x_1 \implies f(x_2)>f(x_1)$, or that
  $f$ is strictly increasing in $[x_1,x_2]$. Since $x_1,x_2$
  are arbitrary, $f$ is strictly increasing in $I$.
\end{solution}
\end{document}
