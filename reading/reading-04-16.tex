\documentclass[boxes, qed]{homework}
\usepackage{enumitem,amsmath,mathtools}
\usepackage{fontspec}
\usepackage{euler}

\defaultfontfeatures{Ligatures=TeX}
% \setmainfont[Ligatures=TeX,Scale=1.25]{[FountainPen.ttf]}

\usepackage{titlesec}

\name{Rohit Wason}
\course{Math 501}
\term{Spring 2021}
\hwnum{(\S{6.4} Taylor's Theorem)}

\newcommand{\bigzero}{\mbox{\normalfont\Large\bfseries 0}}
\newcommand{\rvline}{\hspace*{-\arraycolsep}\vline\hspace*{-\arraycolsep}}

\DeclarePairedDelimiter\abs{\lvert}{\rvert}%
\DeclarePairedDelimiter\norm{\lVert}{\rVert}%

\begin{document}
\begin{problem}Show where Newton's method of finding roots fails.
\end{problem}
\begin{solution}Let $f(x):=\sin{x}$. If we pick $I:=[\pi/2, 3\pi/2]$,
  since $f'(x)=\cos{x}$, it is possible to pick a subinterval $I^* \subset I$ such that
  $\abs{f'(x)}>m=1/2$.\\
  Similarly since $f''(x)=-\sin{x}$ and $\abs{f''(x)}\le M=1$, we can set $K=\frac{M}{2m}=1$.
  If we define a sequence $x_n$ such that:
  $$x_{n+1}:=x_n-\frac{f(x_n)}{f'(x_n)}=x_n-\tan{x_n},$$
  we see that since the value of $\tan{x}$ goes to $\infty$ in $I$
  $x_n$ does not converge.
\end{solution}
\end{document}
