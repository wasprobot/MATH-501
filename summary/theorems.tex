\documentclass{article}
\usepackage[utf8]{inputenc}
\usepackage[english]{babel}
\usepackage{amsmath,amssymb}
\usepackage{mathtools}
\usepackage{amsfonts}
\usepackage{amssymb}
\usepackage{amsthm}
\usepackage[shortlabels]{enumitem}

\newtheorem*{conjecture}{Conjecture}
\newtheorem*{corollary}{Corollary}
\newtheorem*{proposition}{Proposition}
\newtheorem*{theorem}{Theorem}
\newtheorem*{definition}{Definition}
\newtheorem*{lemma}{Lemma}

\newcommand{\bigzero}{\mbox{\normalfont\Large\bfseries 0}}
\newcommand{\rvline}{\hspace*{-\arraycolsep}\vline\hspace*{-\arraycolsep}}

\DeclarePairedDelimiter\abs{\lvert}{\rvert}%
\DeclarePairedDelimiter\norm{\lVert}{\rVert}%

\begin{document}

\begin{theorem}[3.1.5] Let $X=(x_n)$ be a sequence of real numbers, and let $x\in{\mathbb{R}}$.
  The following statements are equivalent.
  \begin{enumerate}[(a)]
    \item $X$ converges to $x$.
    \item For every $\epsilon>0$, there exists a natural number $K$ such that for all $n>K$,
      the terms $x_n$ satisfy $|x_n-x|<\epsilon$.
    \item For every $\epsilon>0$, there exists a natural number $K$ such that for all $n>K$,
      the terms $x-\epsilon<x_n<x+\epsilon$.
    \item For every $\epsilon$-neigborhood $V_{\epsilon}(x)$ or $x$, there exists a natural number
      $K$ such that for all $n\ge{K}$, the terms $x_n$ belong to $V_{\epsilon}(x)$.
  \end{enumerate}
\end{theorem}

\begin{theorem}[3.1.9] Let $X=(x_n:n\in{\mathbb{N}})$ be a sequence of real numbers and let
  $m\in{\mathbb{N}}$. Then the $m$-tail $X_m=\{x_{m+n}:n\in{\mathbb{N}}\}$ of $X$ converges
  if and only if $X$ converges. In this case, $\lim{X_m}=\lim{X}$.
\end{theorem}

\begin{theorem}[3.1.10] Let $(x_n)$ be a sequence of real numbers and let $x\in{\mathbb{R}}$.
  If $(a_n)$ is a sequence of positive real numbers with $\lim{(a_n)}=0$ and if for some $C>0$
  and some $m\in{\mathbb{N}}$ we have
  $\abs{x_n-x} \le Ca_n$ for all $n \ge m$, then it follows that $\lim{(x_n)}=x$.
\end{theorem}

\begin{theorem}[3.2.2]A convergent sequence of real numbers is bounded.
\end{theorem}

\begin{theorem}[3.2.3]
  \begin{enumerate}[(a)]
    \item Let $X=(x_n)$ and $Y=(y_n)$ be sequences of real numbers that converge to $x$ and $y$,
      respectively, and let $c \in{\mathbb{R}}$. Then the sequences  $X+Y,X-Y,X\dot{Y}$, and
      $cX$ converge to $x+y,x-y,xy$ and $cx$, respectively.
    \item If $X=(x_n)$ converges to $x$ and $Z=(z_n)$ is a sequence of nonzero real numbers
      that converges to $z\ne{0}$, then the quotient sequence $X/Z$ converges to $x/z$.
  \end{enumerate}
\end{theorem}

\begin{theorem}[3.2.4] If $X=(x_n)$ is a convergent sequence of real numbers and if $x_n\ge{0}$
  for all $n \in{\mathbb{N}}$, then $x=\lim{(x_n)}\ge{0}$.
\end{theorem}

\begin{theorem}[3.2.5] If $X=(x_n)$ and $Y=(y_n)$ are convergent sequences of real numbers
  and if $x_n\le{y_n}$ for all $n \in{\mathbb{N}}$, then $\lim(x_n)\le{\lim(y_n)}$.
\end{theorem}

\begin{theorem}[3.2.6] If $X=(x_n)$ is a convergent sequence and if $a\le{x_n}\le{b}$ for all
  $n \in{\mathbb{N}}$, then $a\le{\lim(x_n)}\le{b}$.
\end{theorem}

\begin{theorem}[3.2.7] \textbf{Squeeze Theorem} Suppose that $X=(x_n), Y=(y_n)$ and $Z=(z_n)$
  and sequences of real numbers such that $x_n\le{y_n}\le{z_n}$ for all $n \in{\mathbb{N}}$,
  and that $\lim(x_n)=\lim(z_n)$. Then $Y=(y_n)$ is convergent and
  $$\lim(x_n)=\lim(y_n)=\lim(z_n)$$
\end{theorem}

\begin{theorem}[3.2.9] If $x=\lim(x_n)$, then $\abs{x}=\lim(\abs{x_n})$.
\end{theorem}

\begin{theorem}[3.2.10] If $\lim(x_n)=x\ge{0}$, then $\lim(\sqrt{x_n})=\sqrt{x}$.
\end{theorem}

\begin{theorem}[3.2.11] Let $X=(x_n)$ be a sequence of positive real numbers
  such that $L:=\lim(x_{n+1}/x_n)$ exists. If $L<1$, then $(x_n)$ converges and 
  $\lim(x_n)=0$.
\end{theorem}

\begin{theorem}[3.3.2] \textbf{Monotone Convergence Theorem} A monotone sequence of real
  numbers is convergent if and only if it is bounded. Further
  \begin{enumerate}[(a)]
    \item If $X=(x_n)$ is a bounded increasing sequence, then
      $$\lim(x_n)=\sup\{x_n:n \in{\mathbb{N}}\}$$
    \item If $Y=(y_n)$ is a bounded decreasing sequence, then
      $$\lim(y_n)=\inf\{y_n:n \in{\mathbb{N}}\}$$
  \end{enumerate}
\end{theorem}

\begin{theorem}[3.4.2] If a sequence $X=(x_n)$ of real numbers converges to a real 
  number $x$, then any subsequence $X'=(x_{n_k})$ of $X$ also converges to $x$.
\end{theorem}

\begin{theorem}[3.4.4] Let $X=(x_n)$ be a sequence of real numbers. Then following
  are equivalent
  \begin{enumerate}[(i)]
    \item The sequence $X=(x_n)$ does not converge to $x \in{\mathbb{R}}$
    \item There exists an $\epsilon_0>0$ such that for any $k \in{\mathbb{N}}$, there exists
      $n_k \in{\mathbb{N}}$ such that $n_k\ge{k}$ and $\abs{x_{n_k}-x}\ge{\epsilon_0}$
    \item There exists an $\epsilon_0>0$ and a subsequence $X'=(x_{n_k})$ of $X$ such that 
      $\abs{x_{n_k}-x}\ge{\epsilon_0}$ for all $k \in{\mathbb{N}}$
  \end{enumerate}
\end{theorem}

\begin{theorem}[3.4.5] \textbf{Divergence Criteria} If a sequence $X=(x_n)$ of real numbers
  has either of the following prooperties, then $X$ is divergent
  \begin{enumerate}[(i)]
    \item $X$ has two convergent subsequences $X'=(x_{n_k})$ and $X''=(x_{r_k})$ whose limits are not equal.
    \item $X$ is unbounded.
  \end{enumerate}
\end{theorem}

\begin{theorem}[3.4.7] \textbf{Monotone Subsequence Theorem} If $X=(x_n)$ is a sequence of real numbers,
  then there's a subsequence of $X$ that is monotone.
\end{theorem}

\begin{theorem}[3.4.8] \textbf{Bolzano-Weierstraß Theorem} A bounded sequence of real numbers has
  a convergent subsequence.
\end{theorem}

\begin{theorem}[3.4.9] Let $X=(x_n)$ be a bounded sequence of real numbers and let $x \in{\mathbb{R}}$
  have the property that every convergent subsequence of $X$ converges to $x$. Then the sequence
  $X$ converges to $x$.
\end{theorem}

\begin{theorem}[3.4.11] \textbf{Limit Superior and Limit Inferior}\\
  
  If $X=(x_n)$ is a bounded sequence of real numbers, then the following 
  statements for a real number $x^*$ are equivalent
  \begin{enumerate}[(a)]
    \item $x^*=\limsup(x_n)$.
    \item If $\epsilon>0$, there are at most a finite number of $n \in{\mathbb{N}}$ such that
      $x^*+\epsilon<x_n$, but an infinite number of $n \in{\mathbb{N}}$ such that
      $x^*-\epsilon<x_n$.
    \item If $u_m=\sup\{x_n:n\ge{m}\}$, then $x^*=\inf\{u_m:m \in{\mathbb{N}}\}=\lim(u_m)$.
    \item If $S$ is the set of subsequential limits of $(x_n)$, then $x^*=\sup{S}$.
  \end{enumerate}

  Similarly, The following statements are equivalent
  \begin{enumerate}[(a)]
    \item $x^*=\liminf(x_n)$.
    \item If $\epsilon>0$, there are at most a finite number of $n \in{\mathbb{N}}$ such that
      $x^*+\epsilon>x_n$, but an infinite number of $n \in{\mathbb{N}}$ such that
      $x^*-\epsilon>x_n$.
    \item If $u_m=\inf\{x_n:n\le{m}\}$, then $x^*=\sup\{u_m:m \in{\mathbb{N}}\}=\lim(u_m)$.
    \item If $S$ is the set of subsequential limits of $(x_n)$, then $x^*=\inf{S}$.
  \end{enumerate}
\end{theorem}

\begin{theorem}[3.4.12] A bounded sequence $(x_n)$ is convergent if and only if
  $\limsup(x_n)=\liminf(x_n)$.
\end{theorem}

\begin{definition}[3.5.1] A sequence of real numbers, $X=(x_n)$ is said to be a \textbf{Cauchy sequence}
  if for every $\epsilon>0$ there exists as natural number $H(\epsilon)$ such that for all natural
  numbers $n,m\ge{H(\epsilon)}$, the terms $x_n,x_m$ satisfy $\abs{x_n-x_m}<\epsilon$.
\end{definition}

\begin{lemma}[3.5.3] If $X=(x_n)$ is a convergent sequence of real numbers, then $X$ is a Cauchy
  sequence.
\end{lemma}

\begin{lemma}[3.5.4] A Cauchy sequence of real numbers is bounded.
\end{lemma}

\begin{theorem}[3.5.5] A sequence of real numbers is convergent if and only if it is a Cauchy sequence.
\end{theorem}

\begin{definition}[3.5.7] We say that a sequence of real numbers $X=(x_n)$ is \textbf{contractive}
  if there exists a constant $C, 0<C<1$, such that 
  $$\abs{x_{n+2}-x_{n+1}} \le C\abs{x_{n+1}-x_n}$$
\end{definition}

\begin{theorem}[3.5.8] Every contractive sequence is a Cauchy sequence, and hence is convergent.
\end{theorem}

\begin{corollary}[3.5.10] If $X=(x_n)$ is a contractive sequence with constant $C, 0<C<1$,
  and if $x^*:=\lim{X}$ then 
  \begin{enumerate}[(i)]
    \item $\abs{x^*-x_n} \le \frac{C^{n-1}}{1-C} \abs{x_2-x_1}$,
    \item $\abs{x^*-x_n} \le \frac{C}{1-C} \abs{x_n-x_{n-1}}$.
  \end{enumerate}
\end{corollary}

\begin{definition}[3.6.1] Let $X=(x_n)$ be a sequence of real numbers.
  \begin{enumerate}[(i)]
    \item We say that $(x_n)$ \textbf{tends to} $+\infty$, and write $\lim(x_n)=+\infty$, if
      for every $\alpha \in{\mathbb{R}}$ there exists a natural number $K(\alpha)$ such that 
      if $n>K(\alpha)$, then $x_n>\alpha$.
    \item We say that $(x_n)$ \textbf{tends to} $-\infty$, and write $\lim(x_n)=-\infty$, if
      for every $\beta \in{\mathbb{R}}$ there exists a natural number $K(\beta)$ such that 
      if $n>K(\beta)$, then $x_n<\beta$.
  \end{enumerate}
  We say that $(x_n)$ is \textbf{properly divergent} if either $\lim(x_n)=+\infty$ or
  $\lim(x_n)=-\infty$.
\end{definition}

\begin{theorem}[3.6.3] A monotone sequence of real numbers is properly divergent if and only if 
  it is unbounded
  \begin{enumerate}[(a)]
    \item If $(x_n)$ is an unbounded increasing sequence, then $\lim(x_n)=+\infty$.
    \item If $(x_n)$ is an unbounded decreasing sequence, then $\lim(x_n)=-\infty$.
  \end{enumerate}
\end{theorem}

\begin{theorem}[3.6.4] Let $(x_n)$ and $(y_n)$ be two sequence of real numbers and suppose 
  for all $n \in{\mathbb{N}}$
  $$x_n \le y_n$$
  \begin{enumerate}[(a)]
    \item If $\lim(x_n)=+\infty$, then $\lim(y_n)=+\infty$.
    \item If $\lim(y_n)=-\infty$, then $\lim(x_n)=-\infty$.
  \end{enumerate}
\end{theorem}

\begin{theorem}[3.6.5] Let $(x_n)$ and $(y_n)$ be two sequences of positive real numbers 
  and suppose that for some $L \in{\mathbb{R}}, L>0$ we have
  $$\lim(x_n/y_n)=L$$
  Then $\lim(x_n)=+\infty$ if and only if $\lim(y_n)=+\infty$.
\end{theorem}

\begin{theorem}[3.7.3] \textbf{The $n$th Term Test} If the series $\sum{x_n}$ converges,
  then $\lim(x_n)=0$.
\end{theorem}

\begin{theorem}[3.7.4] \textbf{Cauchy Criterion for Series} The series $\sum{x_n}$ converges
  if and only if for every $\epsilon>0$ there exists $M(\epsilon) \in{\mathbb{N}}$ such that 
  if $m\ge n\ge M(\epsilon)$, then
  $$\abs{s_m-s_n} = \abs{x_{n+1}+x_{n+2}+\dots+x_n}<\epsilon$$
\end{theorem}

\begin{theorem}[3.7.7] \textbf{Comparison Test} Let $X=(x_n)$ and $Y=(y_n)$ be sequences of real numbers 
  and suppose that for some $K \in{\mathbb{N}}$ we have
  $0\le x_n\le y_n$ for $n\ge K$.
  \begin{enumerate}[(a)]
    \item Then the convergence of $\sum{y_n}$ implies the convergence of $\sum{x_n}$.
    \item The divergence of $\sum{x_n}$ implies the divergence of $\sum{y_n}$.
  \end{enumerate}
\end{theorem}

\begin{theorem}[3.7.8] \textbf{Limit Comparison Test} Suppose that $X=(x_n)$ and $Y=(y_n)$ are
  strictly positive sequences and suppose that the following limit exists in $\mathbb{R}$
  $$r:=\lim(\frac{x_n}{y_n})$$
  \begin{enumerate}[(a)]
    \item If $r\ne 0$ then $\sum{x_n}$ is convergent if and only if $\sum{y_n}$ is convergent.
    \item If $r=0$ then $\sum{y_n}$ is convergent if and only if $\sum{x_n}$ is convergent.
  \end{enumerate}
\end{theorem}
\end{document}
